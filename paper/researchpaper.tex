\documentclass[12pt]{article}

% Import all needed packages
\usepackage[letterpaper, margin=1in]{geometry}
\usepackage{setspace}

% Set double spacing
\doublespacing

% Bibliography Setup
\usepackage[backend=bibtex]{biblatex}
\addbibresource{bibliography.bib}

% Setup cover page.
\title{Applications of Machine Learning in Enzyme Design}
\author{Jonah Nichols}
\date{\today}

\begin{document}
\maketitle
\section{Methods}
\subsection{Data Acquisition}
RheaDB was utilized to find ester hydrolysis reactions according to the Enzyme Classification
(EC) standards. From selected reactions, Rhea's cross references were utilized
to obtain sequences for the various chemical reactions from UniprotKB. For every
sequence-reaction pair, the Morgan Fingerprint Difference of the reaction was created using
RDKit's built-in fingerprinting function. Due to a lack of known directional data,
unknown directional enzymes were assumed to catalyse the left-to-right version of
the reaction. Next the fingerprint was converted into a PyTorch tensor for import
into the machine learning algorithm.

For every sequence, a one-hot encoding of the sequence was built to create a suitable
output for the reaction. These sequences were then converted into PyTorch tensors
for input into the machine learning algorithm paired up with its corresponding reaction.
These pairs were split randomly into a ratio of 75\% training and 25\% test datasets

\subsection{Neural Network Design}
Given inexperience in the subject, the neural network was designed to a very low
degree of confidence. The input layer was the reaction fingerprint previously generated.
Three repetitions of linear layers followed by ReLU activation layers expanded the
fingerprint out to 7192 bits, corresponding to the longest sequence found. A final
linear layer presented the output of the model.

\subsection{Training and Analysis}
For training, the inputs and outputs were batched to 64 samples and shuffled to
limit overfitting. The L1 Loss function was used to determine the error of the
model. Stochastic gradient descent was utilized to optimize model gradients and
train the model. The training sequence was run for 100 epochs to get an optimized
model.

Due to time constraints, the model's learning capabilities and pattern recognition
were not determined.

\section{Results}
The neural network had an estimated loss of 0.5 (ish).

\section{Discussion}


\end{document}